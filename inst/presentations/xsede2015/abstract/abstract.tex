\documentclass[11pt, a4paper]{article}
\usepackage{amsfonts, amsmath, hanging, hyperref, parskip, times}
\usepackage[numbers]{natbib}
\usepackage[pdftex]{graphicx}
\hypersetup{
  colorlinks,
  linkcolor=blue,
  urlcolor=blue,
  citecolor=blue
}

\let\section=\subsubsection
\newcommand{\pkg}[1]{{\normalfont\fontseries{b}\selectfont #1}} 
\let\proglang=\textit
\let\code=\texttt 
\renewcommand{\title}[1]{\begin{center}{\bf \LARGE #1}\end{center}}
\newcommand{\affiliations}{\footnotesize\centering}
\newcommand{\keywords}{\paragraph{Keywords:}}

\setlength{\topmargin}{-15mm}
\setlength{\oddsidemargin}{-2mm}
\setlength{\textwidth}{165mm}
\setlength{\textheight}{250mm}

\begin{document}
\pagestyle{empty}

\title{XSEDE Text Analytics Gateway}

\begin{center}
  {\bf Mike Black$^1$, Drew Schmidt$^2$}
\end{center}

\begin{affiliations}
1. National Center for Supercomputing Applications, University of Illinois at 
Urbana-Champaign \\[-2pt]
2.National Institute for Computational Sciences, University of Tennessee 
\\[-2pt]
$^1$\underline{mlblac02@gmail.com}\\[-2pt]
$^2$\underline{schmidt@math.utk.edu}\\
\end{affiliations}


\vskip 0.8cm

Increasingly, researchers in the humanities and social sciences are becoming 
more interested in utilizing computational resources to answer their questions. 
 This is particularly relevant for text analysis, which has historically been 
dominated in these fields by qualitative methods.  We believe the resistance to 
quantitative methods is, in part, due to a ``technological familiarity gap'' 
in these fields.

The XSEDE Text Analytics Gateway hopes to be transformational in this regard.  
The goal of the project is to provide researchers with a user-friendly text 
analytics workflow system. Unlike other web-based text mining portals in these 
disciplines such as TaPoR and HTRC, the XSEDE Text Analytics Gateway will not 
merely present users with a list of data retrieval operations. Instead, this 
project implements interactive documentation to walk users through the decisions 
needed to combine data retrieval, preprocessing, analysis, and postprocessing 
into a single, complete workflow. In this respect, the gateway is both a 
research and pedagogical tool. 

The initial audience for the gateway will be researchers with little background 
in computational methods; however, a simpler interface will also be available 
for those with more experience in text analytics who would like to design their 
own workflows independent of the examples provided by the interactive 
documentation. Ideally, these more experienced researchers will be able to use 
this gateway as a place to test and design methodologies in pursuit of larger 
projects. Sample data will be provided for novice users, and the team is 
currently exploring the possibility of integrating access to web APIs from 
relevant data sources for more advanced users.

The core of the gateway’s text analytics functionality will be implemented in 
the popular programming language and analysis package R.  The interface is 
being designed with shiny, an R package that connects R to the web and 
allows for the easy creation of interactive webapps.  The ``interactive 
documentation'' will be created using RMarkdown, which is an enhancement to the 
simple document system markdown.  Elements of the gateway will be embedded 
directly into these markdown documents.

The XSEDE Text Analytics Gateway is currently in the prototyping stage, with 
basic preprocessing and analytics implemented in the non-documentation driven 
interface. Currently, several example case study workflows have been drafted for 
implemented in Markdown. The immediate next steps will be to implement one or 
two sets of interactive documentation, move the framework onto the new XSEDE 
resource Comet, and begin small scale user testing with a sample data set.  In 
this poster, we will share the early developments of the project, including 
technical issues and community building challenges.  The poster should appeal to 
practitioners and gateway developers alike.



%% references: 
\nocite{shiny,rlang, rmd}
\bibliographystyle{chicago}
\bibliography{bib}

\end{document}
